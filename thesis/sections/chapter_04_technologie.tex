\documentclass[../main.tex]{subfiles}

\begin{document}
\section{Wykorzystane technologie}
Z racji charakterystyki tworzonej aplikacji oraz kierując się chęcią skorzystania z gotowych narzędzi, autor dokonał wyboru poniżej opisanych technologii, bibliotek oraz platform programistycznych.

\subsection{Scala}
\textit{Scala} jest silnie typowanym językiem programowania działającym na \textit{Wirtualnej Maszynie Javy}. Język został stworzony w 2001 roku na Politechnice Federalnej w Lozannie przez Martina Odersky'ego. Scala daje możliwość korzystania z dwóch paradygmatów - programowania zorientowanego obiektowo oraz funkcyjnego. Nazwa języka pochodzi od możliwości jego skalowalności \cite{BOOK:ProgrammingInScala}. Zorientowane obiektowo cechy języka pozwalają na łatwe budowanie dużych systemów, natomiast funkcyjne jego części udostępniają możliwość budowania złożonych konstrukcji w elegancki, zwięzły oraz czytelny sposób.

\subsection{Cats}
Biblioteka wprowadzająca do języka \textit{Scala} funkcyjne konstrukcje pozwalające na wykorzystywanie zdecydowanie bardziej zaawansowanych aspektów programowania funkcyjnego niż te istniejące w standardowej bibliotece języka. Biblioteka jest podzielona na rozłączne od siebie moduły, co daje wolność w korzystaniu tylko z potrzebnych konstrukcji. \cite{WEBSITE:CatsDataTypes, BOOK:ScalaWithCats}

\subsection{Akka HTTP}
Biblioteka stworzona przez firmę \textit{Lightbend}\footnote{Lightbend (wcześniej Typesafe) - firma założona przez twórcę języka Scala - Martina Odersky'ego. Dostarcza biblioteki open-source do budowania reaktywnych aplikacji działających na Maszynie Wirtualnej Javy.} dostarczająca narzędzi do tworzenia serwisów http opartych na reaktywnych strumieniach. Zbudowana na bazie innych bibliotek dostarczanych przez tą samą firmę - \textit{Akka Actors} oraz \textit{Akka Stream}. Biblioteka może być używana z powodzeniem zarówno w języku \textit{Scala} jak i \textit{Java}.

\subsection{Circe}
Biblioteka zbudowana z użyciem powyżej wymienionej biblioteki \textit{Cats}. Udostępnia funkcyjny interfejs do pracy z formatem \textit{JSON}\footnote{JavaScript Object Notation} - serializacje obiektów natywnych dla języka Scala do formatu \textit{JSON} oraz deserializację - konwertowanie obiektów zapisanych w formacie \textit{JSON} do konstrukcji którymi można operować w języku \textit{Scala}. Wcześniej biblioteka nosiła nazwę \textit{"JSON for Cats"}.

\subsection{MySQL}
Obecnie jeden z najbardziej popularnych systemów zarządzania relacyjnymi bazami danych\footnote{Według rankingu https://db-engines.com/en/ranking} open-source. Rozwijany przez firmę \textit{Oracle}.

\subsection{Slick}
Biblioteka do pracy z relacyjnymi bazami danych z poziomu kodu języka \textit{Scala}. Pozwala na modelowanie danych, tworzenie zapytań, dla których gwarantuje bezpieczeństwo typów na poziomie kompilacji, modyfikowania oraz tworzenie nowych rekordów. Używa przy tym interfejsu podobnego do kolekcji języka Scala. Należy zauważyć, że \textit{Slick} nie imituje koncepcji ORM\footnote{Object-Relational Mapping - mapowanie obiektowo-relacyjne wykorzystywane w wielu aplikacjach stworzonych paradygmacie zorientowanym obiektowo} \cite{BOOK:EssentialSlick}. Autorzy określają tą bibliotekę jako \textit{Functional Relational Mapping (FRM)}, co tłumaczą interpretowaniem przez \textit{Slick'a} danych jak kolekcji występujących natywnie w języku \cite{WEBSITE:SlickDocs}.

\subsection{ScalaTest}
Narzędzie w znaczny sposób ułatwiające i usprawniające tworzenie jednostkowych testów automatycznych w języku \textit{Scala}. Udostępnia zbiór funkcji sprawdzających własności wyrażeń używając przy tym języka bardzo podobnego do języka mówionego.

\end{document}
