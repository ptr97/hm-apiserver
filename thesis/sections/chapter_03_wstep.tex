\documentclass[../main.tex]{subfiles}

\begin{document}
\section{Wstęp}
\subsection{Cele pracy}

Niniejsza praca ma na celu przybliżenie programowania funkcyjnego - jednego z paradygmatów programowania, jego charakterystyki oraz przedstawienie praktycznego zastosowania tegoż stylu na podstawie stworzenia części serwerowej aplikacji webowej. Tematyka aplikacji powinna stanowić w zamyśle autora jedynie tło, które jest skarbnicą problemów często napotykanych w tego rodzaju aplikacjach. Na podstawie rozwiązywania wyżej wspomnianych problemów, autor ma zamiar przybliżyć czytelnikowi wzorzec funkcyjny oraz opisać jego zalety oraz wady.

\subsection{Tematyka aplikacji}
Aplikacja na podstawie której będziemy poznawać programowanie funkcyjne jest związana z wędrówkami górskimi. Problem jaki został postawiony, to chęć dzielenia się informacjami na temat warunków panujących na szlakach górskich wśród turystów. 
Podstawową funkcjonalnością aplikacji jest udostępnienie użytkownikom dzielenia się opiniami na temat warunków jakie panują w danym miejscu w górach z innymi użytkownikami korzystającymi z aplikacji. Lista lokacji jest zarządzana przez użytkowników będących administratorami.\newline
Opinia jest ściśle skojarzona z użytkownikiem, który ją zamieścił - umożliwia to odfiltrowywanie nieodpowiednich czy też mylących treści oraz wyciąganie konsekwencji wobec ich autorów.\newline
Opinie, które zostaną uznane przez społeczność platformy za nieodpowiednie mogą zostać zgłoszone. Zgłoszenie trafia do administratora, który podejmuje dalsze kroki, włączając w to możliwość zablokowania opinii jak i jej autora.\newline
Natomiast opinie, które zostaną uznane za wyjątkowo pomocne, mają okazje zostać wyróżnione przez użytkowników w sposób znany z popularnych aplikacji społecznościowych. Budowanie bardziej rozbudowanego profilu użytkownika włączając w to informacje na temat przydatności jego porad dla innych to możliwa droga przyszłego rozwoju aplikacji.

\end{document}
